%the full "palette:" 6 7 8 9 10 11 12 14 16 18 21 24 36 48 60 72
%margin paragraphs on the left for 1-sided document
\reversemarginpar
%make sure margin notes look shit-hot \marginpar
%%%fonts %%%%
%\onehalfspacing
% \setstretch{0.1}
\defaultfontfeatures{Ligatures=TeX}
\setmainfont[Numbers={OldStyle,Proportional}]{Adobe Jenson Pro}
\setsansfont{Source Sans Pro}
\setmonofont{Inconsolata}
\setmathfont{Asana Math} 
% Alternatives:
% Latin Modern Math, XITS Math, Neo Euler, Asana Math,
% Lucide New Math, Cambria Math,  Computer Modern


%this is a fallback:
%\setmathfont[range=\mathup/{num,latin,Latin,greek,Greek}]{Adobe Jenson Pro}
%but this is ideal:
\setmathfont[range={\mathup/{num,latin,Latin,greek,Greek},\question,\exclam,\mathdollar,%
  \percent,\ampersand,\lparen,\rparen,\plus,\comma,\period,\mathslash,\mathcolon,%
    \semicolon,\less,\equal,\greater,\backslash,\rbrack,\lbrack,\atsign,\vert,\lbrace,%
    \rbrace,\times,\div}]{Adobe Jenson Pro}
    \setmathfont[range=\mathsf/{num,latin,Latin,greek,Greek}]{Adobe Jenson Pro}
    \setmathfont[range=\mathbfsf/{num,latin,Latin,greek,Greek}]{Adobe Jenson Pro Bold}
    \setmathfont[range=\mathit/{num,latin,Latin,greek,Greek}]%
{Adobe Jenson Pro Italic}
\setmathfont[range=\mathsfit/{num,latin,Latin,greek,Greek}]%
{Adobe Jenson Pro Italic}
\setmathfont[range=\mathbfsfit/{num,latin,Latin,greek,Greek}]%
{Adobe Jenson Pro Bold Italic}
\setmathfont[range=\mathtt->\mathup]{Adobe Jenson Pro}



\pretitle{%
\begin{center}
\fontsize{24}{24}\fontspec[RawFeature={+liga,+dlig,+fina},
                           Style=Alternate]{Adobe Jenson Pro Italic}
}
\posttitle{\par\end{center}\vskip 1em}
\preauthor{\begin{center}
% \fontsize{14}{14}\fontspec[Ligatures=Historical]{Adobe Jenson Pro Italic}
\fontsize{14}{14}\fontspec[RawFeature={+liga,+dlig,+fina},
                          ]{Adobe Jenson Pro Italic}
\lineskip 1em%
\begin{tabular}[t]{c}}
\postauthor{\end{tabular}\par\end{center}}

\predate{\begin{center}
\fontsize{11}{11}\fontspec[LetterSpace=30,
                           Letters=SmallCaps,
                           Numbers={OldStyle,Proportional}]{Adobe Jenson Pro}
}
\postdate{\par\end{center}}


% \newcommand{\sidenote}[1]{
% \marginline{{\fontsize{8}{8}\selectfont
% \begin{spacing}{1}
% #1
% \end{spacing} }}
% }


\postdate{\par\end{center}}


\setkomafont{disposition}{%
\fontspec{Adobe Jenson Pro}
}
  
\setkomafont{section}{%
  \rm\fontsize{11}{11}\fontspec[LetterSpace=20,
                              Letters=SmallCaps]{Adobe Jenson Pro}
}
\setkomafont{subsection}{%
 \fontspec{Adobe Jenson Pro Italic}
}

% \setkomafont{caption}{%

% }
% \setkomafont{captionlabel}{%

% }
% \setkomafont{chapter}{%

% }
% \setkomafont{chapterentry}{%

% }
% \setkomafont{chapterentrypagenumber}{%

% }
% \setkomafont{chapterprefix}{%

% }
% \setkomafont{descriptionlabel}{%

% }
% \setkomafont{dictum}{%

% }
% \setkomafont{dictumauthor}{%

% }
% \setkomafont{dictumtext}{%

% }
% \setkomafont{disposition}{% all sectional unit titles

% }
% \setkomafont{footnote}{% 

% }
% \setkomafont{footnotelabel}{%

% }
% footnotereference
% footnoterule
% labelinglabel
% labelingseparator
% minisec
% pagefoot
% pageheadfoot
% pagenumber
% pagination
% paragraph
% part
% partentry
% partentrypagenumber
% partnumber
% sectionentry
% sectionentrypagenumber
% sectioning %another name for disposition
% subject
% subparagraph
% subsection
% subsubsection
% title

%use \minisec for a heading without TOC or a space, eg top of a recipe

% \documentclass{article}
% \usepackage{fontspec}
% \defaultfontfeatures{Ligatures={Historical}}
% \setmainfont{Adobe Jenson Pro}
% \usepackage{selnolig}
% \nolig{st}{s|t}
% \begin{document}
% best pact 
% \end{document}

%For small caps
%\fontspec[Letters=SmallCaps]{Garamond Premier Pro}
%see also Lining, Monospaced for tables
%\fontspec[Numbers=SlashedZero,OldStyle,Proportional]{Garamond Premier Pro} %
%\fontspec[Contextuals=Swash,Alternate,WordInitial,WordFinal,LineFinal,Inner]{Garamond Premier Pro} %
%\fontspec[StylisticSet=2]{Garamond Premier Pro} %

%individual character variations
%\fontspec[CharacterVariant={5:2}]{EB Garamond Italic}

%style variations
%\fontspec[Style=Alternate]{EB Garamond Italic} % see also Swash,Historic,TitlingCaps,Italic


%\note: use mathscr not mathcal

% OOOh, use 
%\fontspec[Style=Historic]{Adobe Jenson Pro} % looks great

%\fontspec[Style=TitlingCaps]{Adobe Garamond Pro} % looks great

%Go behind fontspec's back
%\fontspec[RawFeature=+smcp;+onum]{Adobe Jenson Pro}

%get maths fonts working?
% \setmathfont{STIXGeneral}
% \setmathfont[range=\mathit/{latin,Latin}]{Sorts Mill Goudy}
% \setmathfont[range=\mathit/{greek,Greek}]{Adobe Jenson Pro}

%define how emphasis is used
\renewcommand\emshape{\bfseries}

% Add feature to a particular shape
%\addfontfeature{ItalicFeatures={Alternate = 1}}

% how to define a new font command which works with \textbf etc.
\newfontfamily\notefont{Adobe Jenson Pro}
\newcommand\textnote[1]{{\notefont #1}} %Double braces are intentional

\renewcommand{\qedsymbol}{$\blacksquare$}

%for a specific face that can't be altered
% \newfontface\fancy[Contextuals={WordInitial,WordFinal}]{Garamond Premier Pro}
%\fontspec[BoldFont={* SemiBold}]{Baskerville}


%\setmainfont[Ligatures={Common,Rare}]{Adobe Caslon Pro}
% \usepackage{microtype}% this needs to be last
